\documentclass{article}

\usepackage[top=0.7in,bottom=0.7in,left=0.7in,right=0.7in]{geometry}
\usepackage{amssymb}	% for \mathbb{}
\usepackage{enumerate}	% for \begin{enumerate}[(a)]

\begin{document}

\title{\vspace{-0.7in}Homework \#1}
\author{Xingyou Song, SID}
\date{}
\maketitle

%%%%%%%%%%%%%%%%%%%%%%%%%%%%%%%%%%%%%%%%%%%%%%%%%%%%%%%%%%%%%%%%%%%%%%%%%%%%%%%%

\section{Question 1}
\begin{enumerate}[(a)]
%%%%%%%%%%%%%%%%%%%%%%%%%%%%%%%%%%%%%%%%%%%%%%%%%%%%%%%%%%%%




%%%%%%%%%%%%%%%%%%%%%%%%%%%%%%%%%%%%%%%%%%%%%%%%%%%%%%%%%%%%



%%%%%%%%%%%%%%%%%%%%%%%%%%%%%%%%%%%%%%%%%%%%%%%%%%%%%%%%%%%%%%%%%%%%%%%%%%%%%%%%

\section{Question 2}

\begin{enumerate}[(a)]
$X_{1} \and X_{2} = 1$ 

\item You can get into the math mode $ax+b=y$ in a sentence.
\item This is a numbered equation, Equation~(\ref{eqn:oneplusone}).
\begin{equation} \label{eqn:oneplusone}
	2^3+1^{-1}=9.
\end{equation}
\item This is an equation without a number.
\begin{displaymath}
	a_1\times b_{11} \leq 1.
\end{displaymath}
\item You can use an equation array to align signs, numbers, or equations: 
\begin{eqnarray*}
	2x+1=3	& \Longrightarrow & 2x=2 \\
			& \Longrightarrow & x=1.
\end{eqnarray*}

\end{enumerate}
You can use $\mathtt{\\newpage}$ to jump to a new page.

\newpage

%%%%%%%%%%%%%%%%%%%%%%%%%%%%%%%%%%%%%%%%%%%%%%%%%%%%%%%%%%%%%%%%%%%%%%%%%%%%%%%%

\section{Question 3: How to Generate Tables}

\begin{enumerate}[(a)]

\item This is a table.
\begin{center}
\begin{tabular}{|c||c|c|}\hline
		& A		& B		\\\hline\hline
	C	& D		& E		\\\hline
	F	& G		& H		\\\hline
\end{tabular}
\end{center}

\item They are more complicated tables (in fact, this is just one table in \LaTeX).
\begin{center}
\begin{tabular}{|c|c|c|c|c|c|c|c|c|c|c|c|c|c|c|c|c|c|c|c|c|c|c|} \cline{1-5}\cline{7-11}\cline{13-17}\cline{19-23}
	\multicolumn{2}{|c|}{$P_1$} & \multicolumn{3}{|c|}{$x$} & & \multicolumn{2}{|c|}{$P_2$} & \multicolumn{3}{|c|}{$x$} & & \multicolumn{2}{|c|}{$P_3$} & \multicolumn{3}{|c|}{$x$} & & \multicolumn{2}{|c|}{$P_4$} & \multicolumn{3}{|c|}{$x$} \\\cline{3-5}\cline{9-11}\cline{15-17}\cline{21-23}
	\multicolumn{2}{|c|}{} & 1 & 2 & 3 & & \multicolumn{2}{|c|}{} & 1 & 2 & 3 & & \multicolumn{2}{|c|}{} & 1 & 2 & 3 & & \multicolumn{2}{|c|}{} & 1 & 2 & 3 \\\cline{1-5}\cline{7-11}\cline{13-17}\cline{19-23}
		& 1 &   &   &   & &     & 1 &   &   &   & &     & 1 &   &   &   & &     & 1 &   &   &   \\\cline{2-5}\cline{8-11}\cline{14-17}\cline{20-23}
	$y$	& 2 &   &   &   & & $y$ & 2 &   &   &   & & $y$ & 2 &   &   &   & & $y$ & 2 &   &   &   \\\cline{2-5}\cline{8-11}\cline{14-17}\cline{20-23}
		& 3 &   &   &   & &     & 3 &   &   &   & &     & 3 &   &   &   & &     & 3 &   &   &   \\\cline{1-5}\cline{7-11}\cline{13-17}\cline{19-23}
\end{tabular}
\end{center}

\end{enumerate}

%%%%%%%%%%%%%%%%%%%%%%%%%%%%%%%%%%%%%%%%%%%%%%%%%%%%%%%%%%%%%%%%%%%%%%%%%%%%%%%%

\section{Some Useful Tips}

\begin{enumerate}[(a)]
\item Notation you may use frequently: $\times \leq \geq \forall \exists \neg \vee \wedge \Longrightarrow \Longleftrightarrow \sum \prod \mathbb{N}$.
\item Improve your writing style:
\begin{center}
\begin{tabular}{|c|c|}\hline
	\textbf{Correct or Better Usage}			& \textbf{Not Suggested	Usage}				\\\hline\hline
	``a'' (check the source file how it comes)	& "a" (check the source file how it comes)	\\\hline
	well-known									& well--known, well---known					\\
	Lines 1--3									& Lines 1-3, Lines 1---3					\\
	UC Berkeley---the best public school		& Berkeley-the, Berkeley--the				\\\hline
	$1,2,\ldots,n$								& $1,2,\cdots,n$							\\
	$1+2+\cdots+n$								& $1+2+\ldots+n$							\\\hline
	$\left(\frac{\frac{1}{3}}{a+b}\right)$		& $(\frac{\frac{1}{3}}{a+b})$				\\\hline
\end{tabular}
\end{center}
\end{enumerate}


%%%%%%%%%%%%%%%%%%%%%%%%%%%%%%%%%%%%%%%%%%%%%%%%%%%%%%%%%%%%%%%%%%%%%%%%%%%%%%%%

\end{document}
